\documentclass[aspectratio = 43, 8pt, a4paper, french]{beamer} % 12pt

% Page format and referencing
% ---------------------------
	%\usepackage[a4paper, total={7in,10.2in}]{geometry}
	%\usepackage[left=1.5cm,right=1.5cm,top=1cm,bottom=2cm]{geometry}
	% \usepackage{fancyhdr}
	%\usepackage[pagebackref,hidelinks]{hyperref}
	%\usepackage[left=1.5cm,right=1.5cm,top=1cm,bottom=2cm]{geometry}

% Language and encoding
% ---------------------
	\usepackage[utf8]{inputenc}
	\usepackage[T1]{fontenc}
	% \usepackage[french]{babel} % english, french, etc.
	\usepackage{lmodern}

% Table of contents
% -----------------
	%\usepackage{tocloft}
	%\usepackage{titlesec}

% Tools and environments
% ---------------------------
	\usepackage{mathtools, soul, commath}
	%\usepackage{ragged2e}
	%\usepackage{thmbox, fancybox}
	\usepackage[underline=false]{thmbox}
	\usepackage{mdframed}

% Symbol libraries
% ----------------
	\usepackage{amsmath, amsfonts, amssymb, latexsym}
	\usepackage{bm, bbm}
	\usepackage{mathalfa, mathrsfs}
	\usepackage[scr]{rsfso}
	%\usepackage{mathabx}
	%\usepackage{textcomp, lmodern}
	
% Graphics, images, tables and plots
% --------------------------
	%\usepackage[x11names]{xcolor}
	% \usepackage{tikz, graphicx, pgfplots}
	% \usepackage[most]{tcolorbox}
	%\usepackage{subfig, wrapfig, multirow, multicol, float}
	%\usepackage{array}
	\usepackage{caption, comment}
	
% Text formatting
% ---------------
	\usepackage{parskip}
	\usepackage[makeroom]{cancel}
	%\usepackage{lipsum}
	
% Code
% ----
	%\usepackage{minted} % command that cancels italic: see https://github.com/gpoore/minted/issues/71
	%\usepackage{etoolbox}
	%\usepackage{newunicodechar}

% New & unsorted
% --------------
	\usepackage{abraces} %other braces
	% \usepackage[lite]{mtpro2} %curly braces
	\usepackage{hhline}
	\usepackage{diagbox}
	\usepackage{esvect}
	\usepackage{scrextend}
	\usepackage{titlesec}
	
	\makeatletter
	\@namedef{ver@framed.sty}{9999/12/31}
	\@namedef{opt@framed.sty}{}
	\makeatother
	
	\usepackage{minted}
	\usemintedstyle{perldoc}
	\usepackage[noline,noend,titlenotnumbered,french]{algorithm2e}
	%\usepackage{framed}
	
	


%\AtEndDocument{\label{lastpage}}
%\lhead{}
%\chead{}
%\rhead{}
%\lfoot{Informatique - MP2I Lycée Fermat}
%\cfoot{}
%\rfoot{\thepage/\pageref{lastpage}}
%\renewcommand{\headrulewidth}{0pt}
%\fancyhfoffset{0.7cm}

% Titles, sections, intros

%	\titleformat*{\section}{\fontsize{18.5pt}{18.5pt}\selectfont\bfseries}
%	\titleformat*{\subsection}{\fontsize{15.5pt}{15.5pt}\selectfont\bfseries}
%	\titleformat*{\subsubsection}{\fontsize{13.25pt}{13.25pt}\selectfont\bfseries}
	
	\renewcommand{\title}[1]{
		\hrule
		\vspace{3mm}
		\begin{center}
			{\Huge{#1}}
		\end{center}
		\vspace{4.5mm}
		\hrule
		\vspace{6mm}
	}
	
	\newcommand{\intro}[1]{
		\begin{center}
			\textit{#1}
			\vspace{4mm}
		\end{center}
	}

% Math formatting, symbols and operators

	\renewcommand{\cal}[1]{\mathcal{#1}}
	\newcommand{\scr}[1]{\mathscr{#1}}
	\newcommand{\bb}[1]{\mathbb{#1}}
	\newcommand{\bbm}[1]{\mathbbm{#1}}
	\renewcommand{\rm}[1]{\mathrm{#1}}
	
	\DeclareMathOperator*{\argmax}{argmax}
	\DeclareMathOperator*{\argmin}{argmin}
	\newcommand{\dotp}{\cdot}
	\newcommand{\x}{\times}
	\DeclareMathOperator{\card}{Card}
	
	\DeclareMathOperator{\Id}{Id}
	%\newcommand{\Id}{I\hspace{-0.5mm}d}
	
	\newcommand{\impright}{\(\bm{(\Rightarrow) :}\) }
	\newcommand{\impleft}{\(\bm{(\Leftarrow) :}\) }
	\renewcommand{\iff}{\Longleftrightarrow}
	
	\newcommand{\fun}[4]{
		\left( \begin{tabular}[h]{rcl}
			\(#1\) & \(\rightarrow\) & \(#2\) \\
			\(#3\) & \(\mapsto\) & \(#4\)
		\end{tabular} \right)
	}
	
	\newcommand{\funn}[6]{
		\left( \begin{tabular}[h]{rcl}
			\(#1\) & \(\rightarrow\) & \(#2\) \\
			\(#3\) & \(\mapsto\) & \(#4\) \\
			\(#5\) & \(\mapsto\) & \(#6\)
		\end{tabular} \right)
	}

	\newcommand{\funnn}[8]{
		\left( \begin{tabular}[h]{rcl}
			\(#1\) & \(\rightarrow\) & \(#2\) \\
			\(#3\) & \(\mapsto\) & \(#4\) \\
			\(#5\) & \(\mapsto\) & \(#6\) \\
			\(#7\) & \(\mapsto\) & \(#8\)
		\end{tabular} \right)
	}
	
	\newcommand{\pbm}[3]{
		\setlength{\tabcolsep}{5.5pt}
		\!\!\!\!\textsf{\textbf{{#1}}} \hspace{-0.8mm}
		\begin{tabular}[t]{|| l }
			\textbf{entrée :\ignorespaces} #2 \\
			\textbf{sortie :\ignorespaces} #3
	\end{tabular}
	}

% Lines, layout and spacing
	
	\newcommand{\vs}[1]{\vspace*{{#1}mm}}
	\newcommand{\hs}[1]{\hspace*{{#1}mm}}
	\newcommand{\nt}{\\[2mm]}
	\newcommand{\nll}{\\[5mm]}
	
	\allowdisplaybreaks
	
	\newcommand{\eqskip}[1]{
		\setlength{\abovedisplayskip}{#1}
		\setlength{\belowdisplayskip}{#1}
	}
	
	\newcommand{\colsep}[1]{\setlength{\tabcolsep}{#1}}

% Bullets and lists
	
	\newcommand{\bdot}{\(\bm{\cdot}\) }
	
	\renewcommand{\i}{\(\bm{i.}\) }
	\newcommand{\ii}{\(\bm{i}\hspace{-0.4mm}\bm{i.}\) }
	\newcommand{\iii}{\(\bm{i}\hspace{-0.4mm}\bm{i}\hspace{-0.4mm}\bm{i.}\) }
	\newcommand{\iv}{\(\bm{i}\hspace{-0.4mm}\bm{v.}\) }
	\renewcommand{\v}{\(\bm{v.}\) }

% Item labels

%	\newenvironment{Definition}[1][]{ \vspace{5mm}
%		\begin{thmbox}[L,leftmargin=8mm,underline=true]
%			{\textbf{Définition
%				\ifthenelse{\equal{#1}{}}{:\ignorespaces}{(\textit{#1}) :\ignorespaces}
%			}}}
%		{\end{thmbox}}
	\newenvironment{Definitions}[1][]{ \vspace{5mm}
		\begin{thmbox}[L,leftmargin=8mm,underline=true]
			{\textbf{Définitions
				\ifthenelse{\equal{#1}{}}{:\ignorespaces}{(\textit{#1}) :\ignorespaces}
			}}}
		{\end{thmbox}}
	
	\newenvironment{Proposition}[1][]{ \vspace{5mm}
		\begin{thmbox}[L,leftmargin=8mm,underline=true]
			{\textbf{Proposition
				\ifthenelse{\equal{#1}{}}{:\ignorespaces}{(\textit{#1}) :\ignorespaces}
			}}}
		{\end{thmbox}}
	
	\newenvironment{Propriete}[1][]{ \vspace{5mm}
		\begin{thmbox}[L,leftmargin=8mm,underline=true]
			{\textbf{Propriété
				\ifthenelse{\equal{#1}{}}{:\ignorespaces}{(\textit{#1}) :\ignorespaces}
			}}}
		{\end{thmbox}}
	\newenvironment{Proprietes}[1][]{ \vspace{5mm}
		\begin{thmbox}[L,leftmargin=8mm,underline=true]
			{\textbf{Propriétés
				\ifthenelse{\equal{#1}{}}{:\ignorespaces}{(\textit{#1}) :\ignorespaces}
			}}}
		{\end{thmbox}}
	
	\newenvironment{Notation}[1][]{ \vspace{5mm}
		\begin{thmbox}[L,leftmargin=8mm,underline=true]
			{\textbf{Notation
				\ifthenelse{\equal{#1}{}}{:\ignorespaces}{(\textit{#1}) :\ignorespaces}
			}}}
		{\end{thmbox}}
	\newenvironment{Notations}[1][]{ \vspace{5mm}
		\begin{thmbox}[L,leftmargin=8mm,underline=true]
			{\textbf{Notations
					\ifthenelse{\equal{#1}{}}{:\ignorespaces}{(\textit{#1}) :\ignorespaces}
		}}}
		{\end{thmbox}}
	
	\newenvironment{Corollaire}[1][]{ \vspace{5mm}
		\begin{thmbox}[L,leftmargin=8mm,underline=true]
			{\textbf{Corollaire
					\ifthenelse{\equal{#1}{}}{:\ignorespaces}{(\textit{#1}) :\ignorespaces}
		}}}
		{\end{thmbox}}
	
	\newenvironment{Lemme}{
		\begin{addmargin}{0mm} \textbf{Lemme :\ignorespaces}}
		{\end{addmargin}}
	
	\newenvironment{Rappel}[1][]{ \vspace{5mm}
		\begin{thmbox}[L,leftmargin=8mm,underline=true]
			{\textbf{Rappel
					\ifthenelse{\equal{#1}{}}{:\ignorespaces}{(\textit{#1}) :\ignorespaces}
		}}}
		{\end{thmbox}}
	
\newenvironment{Preuve}[1][]{
	\begin{addmargin}{5mm} \textbf{Preuve
		\ifthenelse{\equal{#1}{}}{:\ignorespaces}{de #1 :\ignorespaces}} \\}
	{\end{addmargin}}
	
	\newenvironment{Correction}{
		\begin{addmargin}{5mm} \textbf{Correction :\ignorespaces}}
		{\end{addmargin}}

	\newenvironment{Exemple}{
		\begin{addmargin}{5mm} \textbf{Exemple :\ignorespaces}}
		{\end{addmargin}}
	
	\newenvironment{Exemples}{
		\begin{addmargin}{5mm} \textbf{Exemples :\ignorespaces}}
		{\end{addmargin}}
	
	\newenvironment{Illustration}{
		\begin{addmargin}{5mm} \textbf{Illustration :\ignorespaces}}
		{\end{addmargin}}
	
	\newenvironment{Exercice}{
		\begin{addmargin}{5mm} \textbf{Exercice :\ignorespaces}}
		{\end{addmargin}}
	
	\newenvironment{Remarque}{
		\begin{addmargin}{5mm} \textbf{Remarque :\ignorespaces}}
		{\end{addmargin}}
	
	\newenvironment{Aretenir}[1][]{ \vspace{5mm}
		\begin{thmbox}[L,leftmargin=8mm,underline=true]
			{\textbf{\`A retenir
					\ifthenelse{\equal{#1}{}}{:\ignorespaces}{(\textit{#1}) :\ignorespaces}
		}}}
		{\end{thmbox}}
	
	\newenvironment{Syntaxe}[1][]{ \vspace{5mm}
		\begin{thmbox}[L,leftmargin=8mm,underline=true]
			{\textbf{Syntaxe
					\ifthenelse{\equal{#1}{}}{:\ignorespaces}{-- #1 :\ignorespaces}
		}}}
		{\end{thmbox}}
	
	\newenvironment{algo}[4]
		{\colsep{6.5pt} \begin{addmargin}{0mm}\underline{\textbf{Algorithme -- \textsf{#1}\ignorespaces}} \\[0mm]
		\hs{10} \begin{tabular}{||l}
			\begin{algorithm}[H]
				\hs{-4} \begin{tabular}{l}
					\textbf{entrée :} #2
					\ifthenelse{\equal{#3}{}}{}{\\ \textbf{sortie :} #3}
					\ifthenelse{\equal{#4}{}}{}{\\ \textbf{hypothèses :} #4}
				\end{tabular} \\[3mm] \Indp}
		{\end{algorithm} \end{tabular} \end{addmargin}}
	
	\newenvironment{algocont}
	{\begin{addmargin}{0mm}
			\hs{10} \begin{tabular}{||l}
				\begin{algorithm}[H]
					\Indp}
				{\end{algorithm} \end{tabular} \end{addmargin}}
			
	\newenvironment{pscode}[4]
	{\colsep{6.5pt} \begin{addmargin}{0mm}\underline{\textbf{Pseudo-code -- \textsf{#1}\ignorespaces}} #2 \(\to\) #3 \\[0mm]
			\hs{10} \begin{tabular}{||l}
				\begin{algorithm}[H]
					\ifthenelse{\equal{#4}{}}{\vs{2}}{\textbf{hyp :} #4 \\[2mm]} \Indp}
				{\end{algorithm} \end{tabular} \end{addmargin}}

		
% MINTED & CODE
	
	\newcommand{\cc}[1]{\mintinline[escapeinside=@@]{c}{#1}}
	\newcommand{\caml}[1]{\mintinline[escapeinside=@@]{ocaml}{#1}}
	\newcommand{\bash}[1]{\mintinline{bash}{#1}}
	
	\newminted[C]{c}{
		escapeinside=@@,
		tabsize=4,
		linenos
	}

	\newminted[Caml]{ocaml}{
	escapeinside=@@,
	tabsize=4
}
	
% OTHER
	
	\renewcommand{\thealgocf}{}
	\SetInd{1em}{1em}
	
	\newcommand{\entspace}{\hspace*{17.4mm}\ignorespaces}
	\newcommand{\aentspace}{\hspace*{17.4mm}\ignorespaces}
	% \newcommand{\hypspace}{\hspace*{18.5mm}\ignorespaces}
	\newcommand{\listspace}{\hspace*{2.6mm}\ignorespaces}
	\newcommand{\listskip}{\hspace*{5mm} \listspace\ignorespaces}
	
	\renewcommand{\mod}[1]{\,\,\, [\text{mod }#1]}
	
	\newcommand{\codecom}[1]{\colsep{1.2pt}\quad{\footnotesize \textbf{/\!\!/\emph{#1}}}}
	
% OUTDATED

	\newcommand{\plabel}[1]{\textbf{#1}}
	
	\newcommand{\uplabel}[1]{\underline{\textbf{#1}}}
	
	\newcommand{\thuplabel}[2]{
		\vspace{4mm}
		\begin{thmbox}[S,leftmargin=8mm]{\uplabel{#1}}
			{#2}
		\end{thmbox}
		\vspace{2mm}
	}
	
	\newcommand{\shortproof}[1]{\plabel{\textcolor{lightgray}{$\bm{\blacklozenge}$} \textcolor{darkgray}{Preuve : }}{\small \textcolor{darkgray}{#1} \normalsize \nl}}
	
	\newcommand{\corr}[1]{\plabel{\textcolor{lightgray}{$\bm{\blacklozenge}$} \textcolor{darkgray}{Correction :}} \\[2mm] \hspace*{3.5mm} \parbox{167mm}{#1} \nl}

\begin{document}
	
\title{Structures de données séquentielles}

\section{Introduction}
	
	\subsection{Modèle vs. implémentation}
	
		\vs{-2}
		\begin{Definition}[structure de données abstraite]
			Une structure de données abstraite est la description d'un ensemble de données et des opérations que l'on peut y appliquer. \nt
				%
			Elle est donc caractérisée par : \\
				\hs{5} \bdot son type/format, qui répond à la question : ``quel genre d'informations enregistre-t-on ?'' \\
				\hs{5} \bdot ses opérations, qui répondent à la question : ``que peut on faire de ces informations ?''
		\end{Definition}
		
		\vs{2}
		\begin{Exemple}
			Un tableau d'éléments de type \cc{t} est une structure de données abstraite. Elle prend la forme d'une suite finie d'éléments de type \cc{t} et possède les opérations suivantes : \\
				\hs{5} \bdot créer un tableau d'une certaine taille \\
				\hs{5} \bdot accéder à la valeur de l'élément à la position \(i\) dans le tableau \\
				\hs{5} \bdot modifier la valeur de l'élément à la position \(i\) \\
				\hs{5} \bdot obtenir la taille du tableau \nt
				%
			Remarquons bien que caractériser un tableau par un ensemble ne suffirait pas, puisqu'il manquerait les notions d'ordre et de multiplicité. On peut néanmoins munir les ensembles de la multiplicité, on obtient alors ce que l'on appelle des multi-ensembles.
		\end{Exemple}
	
		\begin{Definition}[structure de données concrète]
			Une strcuture de données concrète est une implémentation d'une structure de données abstraite (...).
		\end{Definition}
		
		\vs{2}
		\begin{Exemple}
			L'implémentation suivante des tableaux en C est une structure concrète :
				\begin{C}
									struct tableau{
										int taille;
										int* tab;
									};
				\end{C}
		\end{Exemple}
		
		\vs{3}
		\begin{Remarque}
			La complexité des opérations d'une structure n'est donc fixée que pour une structure concrète, et dépendra pour les structures abstraites de leur implémentation.
		\end{Remarque}
		\vs{2}
		
		\intro{Dans la suite, on dira tout simplement structure de données \\ pour parler de structures de données abstraites.}
	
	\subsection{Opérations}
	
		La description des opérations d'une structure de données ne se veut pas exhaustive ; au contraire, on ne décrit que les opérations élémentaires, qui elles, devront être implémentées par des fonctions de faible complexité.
		
		\vs{2}
		\begin{Exemple}
			Faire la somme des entiers d'un tableau d'éléments de type \cc{int} n'est pas une opération élémentaire pour les tableaux statiques, on ne la précise donc pas.
		\end{Exemple}
		\vs{2}
		
		S'il arrive qu'on fasse toutefois apparaître une opération qui aurait pu se décomposer en opérations plus simples déjà disponible, ce sera toujours pour un gain de complexité.
		
		\vs{2}
		\begin{Exemple}
			Une opération d'initialisation (ou d'ajout, de concaténation...) d'une structure peut être plus efficace si l'on connaît à l'avance \(n\) valeurs à ajouter que si l'on ajoute \(n\) fois une valeur, en faisant les opérations d'ajout une par une.
		\end{Exemple}
	
		\begin{Definitions}[types d'opérations dans une structure]
			Pour une structure de données, on distingue trois sortes d'opérations : \\
				\hs{5} \bdot les constructeurs, qui permettent de créer et/ou d'initialiser la structure \\
				\hs{5} \bdot les accesseurs, qui permettent de récupérer des informations de la structure \\
				\hs{5} \bdot les transformateurs, qui permettent de modifier la structure (en changeant une valeur, \\ \hs{5} \listspace ou même en modifiant la forme de la structure, par exemple pour ajouter ou supprimer \\ \hs{5} \listspace une valeur).
		\end{Definitions}
		\vs{2}
		
		\begin{Remarque}
			On ne considèrera pas de d'opération qui supprime la structure (destructeur).
		\end{Remarque}
		\vs{2}
		
		\intro{Dans la suite de ce chapitre, on s'intéressera surtout à des structures séquentielles \\ d'éléments homogènes. On fixe donc \cc{t} un type pour les éléments.}
		
		
\section{Pile}

	\subsection{Idée}
	
		La structure de pile fonctionne selon le principe ``Last In, First Out'' (ou ``LIFO``) : le premier dernier à rentrer dans la structure doit donc toujours être le premier à en sortir, et inversement le premier à rentrer sera le dernier à sortir.
		
		\vs{2}
		\begin{Illustration}
			On peut se représenter cette structure avec l'image des piles d'assiettes :
		\end{Illustration}
	
	\subsection{Structure de données abstraite}
	
		Une pile représente une suite finie d'éléments de type \cc{t} dont la taille peut varier (au cours du programme, si elle est implémentée). On présente ci-dessous ses opérations élémentaires :
				%
			\begin{center}
				\uplabel{pile}
				\begin{tabular}[t]{|l}
					\bdot \textsf{pile\_vide} : () \(\to\) pile[\cc{t}] \\
					\bdot \textsf{empiler} : \cc{t}, pile[\cc{t}] \(\to\) \(\varnothing\) \\
					\bdot \textsf{dépiler} : pile[\cc{t}] \(\to\) \(\varnothing\) \\
					\bdot \textsf{sommet}$^{(*)}$ : pile[\cc{t}] \(\to\) \cc{t} \\
					\bdot \textsf{est\_vide} : pile[\cc{t}] \(\to\) booléen
				\end{tabular}
			\end{center}
			\vs{3}
			
		\begin{Remarque}
			On ne peut pas demander la taille de la pile (c'est-à-dire on n'y a pas accès).
		\end{Remarque}
		
		\vs{2}
		\begin{Remarque}
			$^{(*)}$Dans certaines définitions du type abstrait de pile, on peut trouver une fonction (souvent nommée \textsf{pop}) qui dépile renvoie de sommet \emph{en le dépilant}, plutôt que d'uniquement le renvoyer : il s'agit donc à la fois d'un modificateur et d'un accesseur.
		\end{Remarque}
	
	\subsection{Implémentation par liste chaînée}
		
		Dans cette implémentation : \\
			\hs{5} \bdot la pile vide sera représentée par un pointeur de valeur \cc{NULL} \\
			\hs{5} \bdot pour empiler, on effectuera un ajout en début de liste \\
			\hs{5} \bdot pour dépiler, on enregistrera \cc{p->next} (la nouvelle première cellule) avant de faire \cc{free(p)}. \nll
			%
		\emph{cf.} TP n°6 pour plus de détails.
		
	\subsection{Implémentation avec un tableau}
	
		On fait à présent l'hypothèse qu'il existe une taille limite \(N_\rm{max}\in\bb{N}^*\) que la pile ne peut pas dépasser. Alors, il est possible d'effectuer une implémentation des piles par un tableau de taille \(N_\rm{max}\) : \vs{2}
			
		\renewcommand{\arraystretch}{1.5}
		\begin{center}
			\begin{tabular}[h]{r|p{0.55\textwidth}}
				Définition de la structure &
				\begin{minipage}[t]{0.5\textwidth}
					\begin{minted}[tabsize=4]{c}
struct s_pile{
	int nb_elem;
	// curseur donnant le sommet de la pile,
	// comme une sorte d'indice de fin
	t* tab;
};

typedef struct s_pile pile;
					\end{minted}
				\end{minipage} \vs{2}\\ \hline
				Fonction \textsf{pile\_vide} &
				\begin{minipage}[t]{0.5\textwidth}
					\begin{minted}[tabsize=4]{c}
pile pile_vide (){
	pile p;
	p.nb_elem = 0;
	p.tab = (t*)malloc(sizeof(t)*Nmax);
	return p;
}
				\end{minted}
			\end{minipage} \vs{2}\\ \hline
						Fonction \textsf{est\_vide} &
		\begin{minipage}[t]{0.5\textwidth}
			\begin{minted}[tabsize=4]{c}
bool est_vide (pile p){
	return p.nb_elem == 0;
}
			\end{minted}
		\end{minipage} \vs{2}\\ \hline
	Fonction \textsf{empiler} &
	\begin{minipage}[t]{0.5\textwidth}
		\begin{minted}[tabsize=4]{c}
void empiler (t val, pile* p){
	// hyp : (*p).nb_elem < Nmax
	if (p->nb_elem == Nmax){
		// message d'erreur
	}
	else{
		p->tab[p->nb_elem] = val;
		p->nb_elem++;
	}
}
		\end{minted}
	\end{minipage} \vs{2}\\ \hline
Fonction \textsf{dépiler} &
\begin{minipage}[t]{0.5\textwidth}
	\begin{minted}[tabsize=4]{c}
void depiler (pile* p){
	// hyp : !(est_vide(*p))
	p->nb_elem--;
}
	\end{minted}
\end{minipage} \vs{2}\\ \hline
Fonction \textsf{sommet} &
\begin{minipage}[t]{0.5\textwidth}
	\begin{minted}[tabsize=4]{c}
t sommet(pile p){
	// hyp : !est_vide(p)
	return p.tab[p.nb_elem-1]
	\end{minted}
\end{minipage}
			\end{tabular}
		\end{center}

\renewcommand{\arraystretch}{1}

\section{File}

	\subsection{Idée}
	
		La structure de file utilise le principe ``First In First Out'' (``FIFO''), ou encore ``Last In Last Out'' (``LILO'') : le premier élément rentré est le premier à sortir, et le dernier rentré est le aussi le dernier à sortir.
		
		\vs{2}
		\begin{Illustration}
			On peut imaginer une queue (une file) de personnes qui entrent dans une pièce par une porte et en ressortent par une autre :
		\end{Illustration}
	
	\subsection{Type abstrait}
	
		Voici les opérations du type abstrait de file :
			%
			\begin{center}
			\uplabel{file}
			\begin{tabular}[t]{|l}
				\bdot \textsf{file\_vide} : () \(\to\) file[\cc{t}] \\
				\bdot \textsf{enfiler} : \cc{t}, file[\cc{t}] \(\to\) \(\varnothing\) \\
				\bdot \textsf{défiler} : file[\cc{t}] \(\to\) \(\varnothing\) \\
				\bdot \textsf{tête}$^{(*)}$ : file[\cc{t}] \(\to\) \cc{t} \\
				\bdot \textsf{est\_vide} : file[\cc{t}] \(\to\) booléen
			\end{tabular}
		\end{center}
		
	\subsection{Implémentation par liste chaînée}
		
		Il est possible d'utiliser des listes doublement chaînées pour implémenter le type abstrait file, mais on peut également se contenter d'une liste simplement chaînée avec : \\
			\hs{5} \bdot un pointeur vers la première cellule, pour défiler (toujours inclus dans une liste chaînée) \\
			\hs{5} \bdot un pointeur vers la dernière cellule, pour enfiler. \nll
			%
		\emph{cf.} TP n°6 pour les détails de l'implémentation.
		
	\subsection{Implémentation par tableau}
	
		Afin de réaliser une implémentation des files avec des tableaux, il nous faut encore une fois nous placer sous certaines hypothèses : le nombre d'insertions est au plus \(N_\rm{max}\) où \(N_\rm{max} \in \bb{N}^*\).
		
		\begin{center}
			\begin{tabular}[h]{r|p{0.6\textwidth}}
				Définition de la structure &
				\begin{minipage}[t]{0.5\textwidth}
					\begin{minted}[tabsize=4]{c}
struct file{
	int debut;
	int fin;
	t* tab;
};
					\end{minted}
				\end{minipage}
				\end{tabular}
			\end{center}
		
		\vs{2}
		\emph{cf.} TP n°6 pour le reste (définitions des fonctions réalisant les opérations élémentaires).
		
	\subsection{Implémentation avec deux piles}
		
		Le principe de l'implémentation des files avec deux piles est le suivant : \\
			\hs{5} \bdot on se donne deux piles, \(p_1\) et \(p_2\) \\
			\hs{5} \bdot pour enfiler un élément, on l'empile sur \(p_1\) \\
			\hs{5} \bdot pour défiler, on dépile l'élément au sommet de \(p_2\) lorsque celle-ci est non vide, et dans le cas contraire, on renverse \(p_1\) puis on transvase le contenu renversé dans \(p_2\) afin de pouvoir y dépiler l'élément à défiler. \nt
			%
		On illustre ci-dessous cette implémentation : \nll
			%
		On donne maintenant le pseudo-code des opérations élémentaires :
		
		\begin{pscode}{file\_vide}{()}{file}{}
			\(F.p_1 = \) \textsf{pile\_vide}() \\
			\(F.p_2 = \) \textsf{pile\_vide}() \\
			Renvoyer \(F\)
		\end{pscode}
		
		\vs{1}
		\begin{pscode}{\(\bm{F.}\)est\_vide}{()}{booléen}{}
			Retourner (\(F.p_1.\)\textsf{est\_vide} et \(F.p_2.\)\textsf{est\_vide})
		\end{pscode}
		
		\vs{1}
		\begin{pscode}{\(\bm{F.}\)tête}{()}{\cc{elem}}{\(F\) est non vide}
			Si \(F.p_2.\)\textsf{est\_vide}() alors \\ \Indp
				Tant que (non \(F.p_1.\)\textsf{est\_vide}) \\ \Indp
					\(F.p_2.\)\textsf{empile}(\(F.p_1.\)\textsf{sommet}) \\
					\(F.p_1.\)\textsf{dépile}() \\ \Indm \Indm
			Renvoyer \(F.p_2.\)\textsf{sommet}()
		\end{pscode}
		
		\vs{1}
		\begin{pscode}{\(\bm{F.}\)\textsf{enfile}}{(\cc{elem} \(e\))}{()}{}
			\(F.p_1.\)\textsf{empile}\((e)\)
		\end{pscode}
		
		\vs{1}
		\begin{pscode}{\(\bm{F.}\)\textsf{défile}}{()}{()}{\(F\) est non vide}
			Si \(F.p_2.\)\textsf{est\_vide}() alors \\ \Indp
				Tant que (non \(F.p_1.\)\textsf{est\_vide}) \\ \Indp
					\(F.p_2.\)\textsf{empile}(\(F.p_1.\)\textsf{sommet}) \\
					\(F.p_1.\)\textsf{dépiler}() \\ \Indm \Indm
			\(F.p_2.\)\textsf{dépile}()
		\end{pscode}
	
		\vs{4}
		\begin{Remarque}
			On a ici utilisé ce qu'on appelle la notation objet pour les fonctions.
		\end{Remarque}
	
\section{Listes}

	\subsection{Définition}
		
		\vs{-2}
		\begin{Definition}[liste]
			Une liste est une structure de données abstraite permettant de stocker une suite finie d'éléments (munis d'un ordre et d'une multiplicité), dans laquelle on peut réaliser des insertions et suppressions en début, milieu ou fin de la séquence, et que l'on peut parcourir.
		\end{Definition}
		\vs{2}
		
		La liste doit être munie des opérations suivantes :
			%
			\begin{center}
				\uplabel{liste}
				\begin{tabular}[t]{|l}
					\underline{constructeur} \\
					\bdot \textsf{liste\_vide} : () \(\to\) liste \\[1mm]
					\underline{accesseurs} \\
					\bdot \textsf{\(L.\)est\_vide} : () \(\to\) booléen \\
					\bdot \textsf{\(L.\)début} : () \(\to\) place$^{(*)}$ \\
					\bdot \textsf{\(L.\)suivant} : place \(p\) \(\to\) place \\
					\bdot \textsf{\(L.\)contenu} : place \(p\) \(\to\) \cc{elem}
				\end{tabular}
			\end{center}
			
			\begin{center}
				\hs{25}
				\begin{tabular}[t]{|l}
					\bdot \textsf{\(L.\)est\_dernier} : place \(p\) \(\to\) booléen \\[1mm]
					\underline{transformateurs} \\
					\bdot \textsf{\(L.\)ajoute\_après} : place \(p\), \cc{elem} \(e\) \(\to\) \(\varnothing\) \\
					\bdot \textsf{\(L.\)ajoute\_début} : \cc{elem} \(e\) \(\to\) \(\varnothing\) \\
					\bdot \textsf{\(L.\)supprime} : place \(p\) \(\to\) \(\varnothing\)
				\end{tabular}
			\end{center}
		
		\vs{3}
		\begin{Remarque}
			$^{(*)}$Les positions sont ici à voir comme des pointeurs, et non pas des numéros.
		\end{Remarque}
	
	\subsection{Implémentation par listes doublement chaînées}
	
		L'implémentation par tableau est possible mais elle est restrictive (la taille de la liste est bornée) et de plus, l'insertion et la suppression d'éléments serait en temps linéaire. \nll
			%
		On préfèrera donc implémenter les listes à l'aide de listes doublement chaînées : chaque cellule possède, en plus d'un champ contenant un élément de type \cc{elem} et du pointeur vers la cellule suivante, un pointeur vers la cellule précédente pour remonter en arrière. \nll
			%
		\emph{cf.} TP n°6 pour l'implémentation.
		
	\subsection{Fonctions non élémentaires}
		
		On propose ci-dessous le pseudo-code de deux fonctions non élémentaires pour la structure, mais néanmoins très utiles lorsque l'on manipule des listes : \\
			\hs{5} \bdot \textsf{\(L.\)est\_présent}, qui teste si un élément \(e\) est présent dans \(L\) \\
			\hs{5} \bdot \textsf{\(L.\)première\_place}, qui retourne la première place (occurence) de \(e\) dans \(L\) en supposant qu'elle \\ \listskip existe
			
		\begin{pscode}{\(\bm{L.}\)est\_présent}{(\cc{elem} \(e\))}{booléen}{}
			Si \(L.\)\textsf{est\_vide}() alors \\ \Indp
				renvoyer Faux \\ \Indm
			Sinon \\ \Indp
				place \(p\) = \(L.\)\textsf{début} \\
				Tant que (non \(L.\)\textsf{est\_dernier}\((p)\)) \\ \Indp
					Si \(L.\)\textsf{contenu}\((p) = e\) alors \\ \Indp
						renvoyer Vrai \\ \Indm
					Sinon \\ \Indp
						\(p \gets L.\)\textsf{suivant}\((p)\) \\ \Indm \Indm
				Renvoyer (\(L.\)\textsf{contenu}\((p) = e\))
		\end{pscode}
		
		\vs{3}
		\begin{pscode}{\(\bm{L.}\)première\_place}{(\cc{elem} \(e\))}{place}{\(L.\)\textsf{est\_présent}$(e)$}
			place \(p\) = \(L.\)\textsf{début} \\
			Tant que (non \(L.\)\textsf{est\_dernier}\((p)\)) \\ \Indp
				Si \(L.\)\textsf{contenu}\((p) = e\) alors \\ \Indp
					renvoyer \(p\) \\ \Indm
				Sinon \\ \Indp
				\(p \gets L.\)\textsf{suivant}\((p)\) \\ \Indm \Indm
			Renvoyer \(p\)
		\end{pscode}
	
\section{Complexité amortie}

	Pour une structure de données donnée et pour une opération donnée sur cette structure, considérer la complexité pire cas de cette opération n'est pas toujours pertinent puisque le coût du pire cas induit celui des opérations qui améliorent l'état de la structure. \nt
		%
	Pour pallier ce manque de représentativité, nous étudierons la complexité amortie, c'est-à-dire la complexité globale de plusieurs opérations successives.
	
	\vs{2}
	\begin{Remarque}
		Il faut distinguer la complexité amortie, qui est une moyenne sur des appels successifs, de la complexité dite moyenne, qui correspond plutôt à une moyenne sur les différentes instances possibles pour un seul appel.
	\end{Remarque}

	\subsection{Table dynamique}
	
		\intro{La table dynamique est un exemple illustrant la nécessité de considérer la complexité amortie pour estimer fidèlement le coùt d'une opération. Dans cette section, nous définissons cette structure et faisons une étude précise de complexité amortie pour l'une de ses opérations.}
		
		\vs{-2}
		La table dynamique, contrairement aux tableaux statiques, est une structure dont la taille peut varier : on peut ajouter des éléments tant qu'il reste de la place, et une fois que la table est remplie, la prochaine insertion fera doubler sa taille en vue d'accueillir des nouveaux éléments par la suite.
			\begin{center}
				\uplabel{table dynamique}
				\begin{tabular}[t]{|l}
					\underline{constructeur} \\
					\bdot \textsf{crée\_case} : () \(\to\) table dynamique \\[1mm]
					\underline{accesseurs} \\
					\bdot \textsf{\(T.\)élément} : \cc{int} \(i\) \(\to\) \cc{elem} \\
					\bdot \textsf{\(T.\)nb\_elem} : () \(\to\) \cc{int} \\[1mm]
					\underline{transformateurs} \\
					\bdot \textsf{\(T.\)modifie} : \cc{int} \(i\), \cc{elem} \(e\) \(\to\) \(\varnothing\) \\
					\bdot \textsf{\(T.\)ajoutefin} : \cc{elem} \(e\) \(\to\) \(\varnothing\) \\
					\bdot \textsf{\(T.\)retirefin} : () \(\to\) \(\varnothing\)
				\end{tabular}
			\end{center}
			\vs{3}
		
		Soit \(N\in\bb{N}^*\). On considère la suite d'appels ci-dessous : \nt
			\hs{8.5} \begin{tabular}[h]{||l}
				\begin{algorithm}[H]
					\(T = \) \textsf{crée\_case}\(()\) \\
					Pour \(i\) allant de \(1\) à \(N\) \\ \Indp
						\(T.\)\textsf{ajoutefin}\((i)\)
				\end{algorithm}
			\end{tabular}
			
		Pour \(j\in[1..N]\), on note \(\sigma_j\) le coût du \(j\)-ième appel à \textsf{ajoutefin}. \\ 
		Alors, on peut représenter comme ci-dessous l'état (taille, cases vides et pleines) de la table \(T\) à chaque appel, c'est-à-dire à chaque tour de la boucle ``Pour'' : \nll
		
		\colsep{1.5pt}
		\begin{Propriete}
			On en déduit alors : \(\displaystyle \forall\,j\in[1..N],\, \sigma_j =
				\left\{ \begin{tabular}[h]{l}
					\(1\) si \(j-1\notin \{2^k\,|\,k\in \bb{N}\}\) \\
					\(2+2^k\) s'il existe \(k\in\bb{N}\) tel que \(j-1=2^k\)
				\end{tabular}\right.
			\) \\[1mm]
			soit encore, en changeant d'indice : \(\displaystyle \forall\,j'\in[0..N-1],\, \sigma_{j'+1} =
			\left\{ \begin{tabular}[h]{l}
				\(2+2^k\) si \(j' = 2^k\) pour un \(k\in\bb{N}\) \\
				\(1\) sinon
			\end{tabular}\right.
			\).
		\end{Propriete}
		
		\eqskip{3mm}
		\vs{2}
		Le coût total vaut alors : \(S_N = \displaystyle\sum_{j=1}^N \sigma_j = \sum_{j=0}^{N-1} \sigma_{j+1}\). \\[1mm]
		Notons \(D = 2^{\lfloor \log_2(N-1) \rfloor}\) alors, \(D\) est la plus grande puissance de 2 inférieure ou égale à \(N-1\). \\
		Ainsi, \(\forall\,j \in [D+1,N]\), \(\sigma^{j+1} = 1\) car \(j\) n'est pas une puissance de 2. En notant \(p = \log_2(D)\), on a :
			\begin{align*}
				S_N & = \sum_{j=0}^{D-1} \sigma_{j+1} + \sigma_{D+1} + \sum_{j=D+1}^{N-1} \sigma_{j+1} \\
				& = \bigg(\underbrace{\sigma_1}_{=1} + \sum_{k=0}^{p-1} \sum_{j=2^k}^{2^{k+1}-1} \sigma_{j+1}\bigg) + \underbrace{\sigma_{D+1}}_{=2+2^p} + (N-1 - (D+1)+1) \\
				& = (3 + D) + (N - D - 1) + \sum_{k=0}^{p-1}\bigg(\sigma_{2^k+1} + \sum_{j=2^k+1}^{2^{k+1}-1}\sigma_{j+1}\bigg) \\
				& = N +2 + \sum_{k=0}^{p-1} (2^k + 2) + (2^{k+1} - 2^{k}-1) \\
				& = N +2 + \sum_{k=0}^{p-1} 2^{k+1} + 1 \\
				& = N + p + 2\sum_{k=0}^{p-1} 2^k \\
				& = N+2 + p + 2(2^{p} - 1) \\
				& = N + p + 2D \\
				& = N + \lfloor \log_2(N-1)\rfloor + 2^{\lfloor \log_2(N-1)\rfloor + 1}
			\end{align*}
		Or, \(\lfloor \log_2(N-1)\rfloor\in\Theta(\ln(N))\) et \(2^{\lfloor \log_2(N-1)\rfloor +1} \in \Theta(N)\).
		
		\begin{Propriete}
			On a donc finalement \(S_N \in \Theta(N)\) : la complexité amortie d'une succession opérations d'ajout dans une table dynamique est linéaire en le nombre d'ajouts \(N\).
		\end{Propriete}
	
\end{document}