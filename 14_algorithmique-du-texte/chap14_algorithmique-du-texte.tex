\documentclass[aspectratio = 43, 8pt, a4paper, french]{beamer} % 12pt

% Page format and referencing
% ---------------------------
	%\usepackage[a4paper, total={7in,10.2in}]{geometry}
	%\usepackage[left=1.5cm,right=1.5cm,top=1cm,bottom=2cm]{geometry}
	% \usepackage{fancyhdr}
	%\usepackage[pagebackref,hidelinks]{hyperref}
	%\usepackage[left=1.5cm,right=1.5cm,top=1cm,bottom=2cm]{geometry}

% Language and encoding
% ---------------------
	\usepackage[utf8]{inputenc}
	\usepackage[T1]{fontenc}
	% \usepackage[french]{babel} % english, french, etc.
	\usepackage{lmodern}

% Table of contents
% -----------------
	%\usepackage{tocloft}
	%\usepackage{titlesec}

% Tools and environments
% ---------------------------
	\usepackage{mathtools, soul, commath}
	%\usepackage{ragged2e}
	%\usepackage{thmbox, fancybox}
	\usepackage[underline=false]{thmbox}
	\usepackage{mdframed}

% Symbol libraries
% ----------------
	\usepackage{amsmath, amsfonts, amssymb, latexsym}
	\usepackage{bm, bbm}
	\usepackage{mathalfa, mathrsfs}
	\usepackage[scr]{rsfso}
	%\usepackage{mathabx}
	%\usepackage{textcomp, lmodern}
	
% Graphics, images, tables and plots
% --------------------------
	%\usepackage[x11names]{xcolor}
	% \usepackage{tikz, graphicx, pgfplots}
	% \usepackage[most]{tcolorbox}
	%\usepackage{subfig, wrapfig, multirow, multicol, float}
	%\usepackage{array}
	\usepackage{caption, comment}
	
% Text formatting
% ---------------
	\usepackage{parskip}
	\usepackage[makeroom]{cancel}
	%\usepackage{lipsum}
	
% Code
% ----
	%\usepackage{minted} % command that cancels italic: see https://github.com/gpoore/minted/issues/71
	%\usepackage{etoolbox}
	%\usepackage{newunicodechar}

% New & unsorted
% --------------
	\usepackage{abraces} %other braces
	% \usepackage[lite]{mtpro2} %curly braces
	\usepackage{hhline}
	\usepackage{diagbox}
	\usepackage{esvect}
	\usepackage{scrextend}
	\usepackage{titlesec}
	
	\makeatletter
	\@namedef{ver@framed.sty}{9999/12/31}
	\@namedef{opt@framed.sty}{}
	\makeatother
	
	\usepackage{minted}
	\usemintedstyle{perldoc}
	\usepackage[noline,noend,titlenotnumbered,french]{algorithm2e}
	%\usepackage{framed}
	
	


%\AtEndDocument{\label{lastpage}}
%\lhead{}
%\chead{}
%\rhead{}
%\lfoot{Informatique - MP2I Lycée Fermat}
%\cfoot{}
%\rfoot{\thepage/\pageref{lastpage}}
%\renewcommand{\headrulewidth}{0pt}
%\fancyhfoffset{0.7cm}

% Titles, sections, intros

%	\titleformat*{\section}{\fontsize{18.5pt}{18.5pt}\selectfont\bfseries}
%	\titleformat*{\subsection}{\fontsize{15.5pt}{15.5pt}\selectfont\bfseries}
%	\titleformat*{\subsubsection}{\fontsize{13.25pt}{13.25pt}\selectfont\bfseries}
	
	\renewcommand{\title}[1]{
		\hrule
		\vspace{3mm}
		\begin{center}
			{\Huge{#1}}
		\end{center}
		\vspace{4.5mm}
		\hrule
		\vspace{6mm}
	}
	
	\newcommand{\intro}[1]{
		\begin{center}
			\textit{#1}
			\vspace{4mm}
		\end{center}
	}

% Math formatting, symbols and operators

	\renewcommand{\cal}[1]{\mathcal{#1}}
	\newcommand{\scr}[1]{\mathscr{#1}}
	\newcommand{\bb}[1]{\mathbb{#1}}
	\newcommand{\bbm}[1]{\mathbbm{#1}}
	\renewcommand{\rm}[1]{\mathrm{#1}}
	
	\DeclareMathOperator*{\argmax}{argmax}
	\DeclareMathOperator*{\argmin}{argmin}
	\newcommand{\dotp}{\cdot}
	\newcommand{\x}{\times}
	\DeclareMathOperator{\card}{Card}
	
	\DeclareMathOperator{\Id}{Id}
	%\newcommand{\Id}{I\hspace{-0.5mm}d}
	
	\newcommand{\impright}{\(\bm{(\Rightarrow) :}\) }
	\newcommand{\impleft}{\(\bm{(\Leftarrow) :}\) }
	\renewcommand{\iff}{\Longleftrightarrow}
	
	\newcommand{\fun}[4]{
		\left( \begin{tabular}[h]{rcl}
			\(#1\) & \(\rightarrow\) & \(#2\) \\
			\(#3\) & \(\mapsto\) & \(#4\)
		\end{tabular} \right)
	}
	
	\newcommand{\funn}[6]{
		\left( \begin{tabular}[h]{rcl}
			\(#1\) & \(\rightarrow\) & \(#2\) \\
			\(#3\) & \(\mapsto\) & \(#4\) \\
			\(#5\) & \(\mapsto\) & \(#6\)
		\end{tabular} \right)
	}

	\newcommand{\funnn}[8]{
		\left( \begin{tabular}[h]{rcl}
			\(#1\) & \(\rightarrow\) & \(#2\) \\
			\(#3\) & \(\mapsto\) & \(#4\) \\
			\(#5\) & \(\mapsto\) & \(#6\) \\
			\(#7\) & \(\mapsto\) & \(#8\)
		\end{tabular} \right)
	}
	
	\newcommand{\pbm}[3]{
		\setlength{\tabcolsep}{5.5pt}
		\!\!\!\!\textsf{\textbf{{#1}}} \hspace{-0.8mm}
		\begin{tabular}[t]{|| l }
			\textbf{entrée :\ignorespaces} #2 \\
			\textbf{sortie :\ignorespaces} #3
	\end{tabular}
	}

% Lines, layout and spacing
	
	\newcommand{\vs}[1]{\vspace*{{#1}mm}}
	\newcommand{\hs}[1]{\hspace*{{#1}mm}}
	\newcommand{\nt}{\\[2mm]}
	\newcommand{\nll}{\\[5mm]}
	
	\allowdisplaybreaks
	
	\newcommand{\eqskip}[1]{
		\setlength{\abovedisplayskip}{#1}
		\setlength{\belowdisplayskip}{#1}
	}
	
	\newcommand{\colsep}[1]{\setlength{\tabcolsep}{#1}}

% Bullets and lists
	
	\newcommand{\bdot}{\(\bm{\cdot}\) }
	
	\renewcommand{\i}{\(\bm{i.}\) }
	\newcommand{\ii}{\(\bm{i}\hspace{-0.4mm}\bm{i.}\) }
	\newcommand{\iii}{\(\bm{i}\hspace{-0.4mm}\bm{i}\hspace{-0.4mm}\bm{i.}\) }
	\newcommand{\iv}{\(\bm{i}\hspace{-0.4mm}\bm{v.}\) }
	\renewcommand{\v}{\(\bm{v.}\) }

% Item labels

%	\newenvironment{Definition}[1][]{ \vspace{5mm}
%		\begin{thmbox}[L,leftmargin=8mm,underline=true]
%			{\textbf{Définition
%				\ifthenelse{\equal{#1}{}}{:\ignorespaces}{(\textit{#1}) :\ignorespaces}
%			}}}
%		{\end{thmbox}}
	\newenvironment{Definitions}[1][]{ \vspace{5mm}
		\begin{thmbox}[L,leftmargin=8mm,underline=true]
			{\textbf{Définitions
				\ifthenelse{\equal{#1}{}}{:\ignorespaces}{(\textit{#1}) :\ignorespaces}
			}}}
		{\end{thmbox}}
	
	\newenvironment{Proposition}[1][]{ \vspace{5mm}
		\begin{thmbox}[L,leftmargin=8mm,underline=true]
			{\textbf{Proposition
				\ifthenelse{\equal{#1}{}}{:\ignorespaces}{(\textit{#1}) :\ignorespaces}
			}}}
		{\end{thmbox}}
	
	\newenvironment{Propriete}[1][]{ \vspace{5mm}
		\begin{thmbox}[L,leftmargin=8mm,underline=true]
			{\textbf{Propriété
				\ifthenelse{\equal{#1}{}}{:\ignorespaces}{(\textit{#1}) :\ignorespaces}
			}}}
		{\end{thmbox}}
	\newenvironment{Proprietes}[1][]{ \vspace{5mm}
		\begin{thmbox}[L,leftmargin=8mm,underline=true]
			{\textbf{Propriétés
				\ifthenelse{\equal{#1}{}}{:\ignorespaces}{(\textit{#1}) :\ignorespaces}
			}}}
		{\end{thmbox}}
	
	\newenvironment{Notation}[1][]{ \vspace{5mm}
		\begin{thmbox}[L,leftmargin=8mm,underline=true]
			{\textbf{Notation
				\ifthenelse{\equal{#1}{}}{:\ignorespaces}{(\textit{#1}) :\ignorespaces}
			}}}
		{\end{thmbox}}
	\newenvironment{Notations}[1][]{ \vspace{5mm}
		\begin{thmbox}[L,leftmargin=8mm,underline=true]
			{\textbf{Notations
					\ifthenelse{\equal{#1}{}}{:\ignorespaces}{(\textit{#1}) :\ignorespaces}
		}}}
		{\end{thmbox}}
	
	\newenvironment{Corollaire}[1][]{ \vspace{5mm}
		\begin{thmbox}[L,leftmargin=8mm,underline=true]
			{\textbf{Corollaire
					\ifthenelse{\equal{#1}{}}{:\ignorespaces}{(\textit{#1}) :\ignorespaces}
		}}}
		{\end{thmbox}}
	
	\newenvironment{Lemme}{
		\begin{addmargin}{0mm} \textbf{Lemme :\ignorespaces}}
		{\end{addmargin}}
	
	\newenvironment{Rappel}[1][]{ \vspace{5mm}
		\begin{thmbox}[L,leftmargin=8mm,underline=true]
			{\textbf{Rappel
					\ifthenelse{\equal{#1}{}}{:\ignorespaces}{(\textit{#1}) :\ignorespaces}
		}}}
		{\end{thmbox}}
	
\newenvironment{Preuve}[1][]{
	\begin{addmargin}{5mm} \textbf{Preuve
		\ifthenelse{\equal{#1}{}}{:\ignorespaces}{de #1 :\ignorespaces}} \\}
	{\end{addmargin}}
	
	\newenvironment{Correction}{
		\begin{addmargin}{5mm} \textbf{Correction :\ignorespaces}}
		{\end{addmargin}}

	\newenvironment{Exemple}{
		\begin{addmargin}{5mm} \textbf{Exemple :\ignorespaces}}
		{\end{addmargin}}
	
	\newenvironment{Exemples}{
		\begin{addmargin}{5mm} \textbf{Exemples :\ignorespaces}}
		{\end{addmargin}}
	
	\newenvironment{Illustration}{
		\begin{addmargin}{5mm} \textbf{Illustration :\ignorespaces}}
		{\end{addmargin}}
	
	\newenvironment{Exercice}{
		\begin{addmargin}{5mm} \textbf{Exercice :\ignorespaces}}
		{\end{addmargin}}
	
	\newenvironment{Remarque}{
		\begin{addmargin}{5mm} \textbf{Remarque :\ignorespaces}}
		{\end{addmargin}}
	
	\newenvironment{Aretenir}[1][]{ \vspace{5mm}
		\begin{thmbox}[L,leftmargin=8mm,underline=true]
			{\textbf{\`A retenir
					\ifthenelse{\equal{#1}{}}{:\ignorespaces}{(\textit{#1}) :\ignorespaces}
		}}}
		{\end{thmbox}}
	
	\newenvironment{Syntaxe}[1][]{ \vspace{5mm}
		\begin{thmbox}[L,leftmargin=8mm,underline=true]
			{\textbf{Syntaxe
					\ifthenelse{\equal{#1}{}}{:\ignorespaces}{-- #1 :\ignorespaces}
		}}}
		{\end{thmbox}}
	
	\newenvironment{algo}[4]
		{\colsep{6.5pt} \begin{addmargin}{0mm}\underline{\textbf{Algorithme -- \textsf{#1}\ignorespaces}} \\[0mm]
		\hs{10} \begin{tabular}{||l}
			\begin{algorithm}[H]
				\hs{-4} \begin{tabular}{l}
					\textbf{entrée :} #2
					\ifthenelse{\equal{#3}{}}{}{\\ \textbf{sortie :} #3}
					\ifthenelse{\equal{#4}{}}{}{\\ \textbf{hypothèses :} #4}
				\end{tabular} \\[3mm] \Indp}
		{\end{algorithm} \end{tabular} \end{addmargin}}
	
	\newenvironment{algocont}
	{\begin{addmargin}{0mm}
			\hs{10} \begin{tabular}{||l}
				\begin{algorithm}[H]
					\Indp}
				{\end{algorithm} \end{tabular} \end{addmargin}}
			
	\newenvironment{pscode}[4]
	{\colsep{6.5pt} \begin{addmargin}{0mm}\underline{\textbf{Pseudo-code -- \textsf{#1}\ignorespaces}} #2 \(\to\) #3 \\[0mm]
			\hs{10} \begin{tabular}{||l}
				\begin{algorithm}[H]
					\ifthenelse{\equal{#4}{}}{\vs{2}}{\textbf{hyp :} #4 \\[2mm]} \Indp}
				{\end{algorithm} \end{tabular} \end{addmargin}}

		
% MINTED & CODE
	
	\newcommand{\cc}[1]{\mintinline[escapeinside=@@]{c}{#1}}
	\newcommand{\caml}[1]{\mintinline[escapeinside=@@]{ocaml}{#1}}
	\newcommand{\bash}[1]{\mintinline{bash}{#1}}
	
	\newminted[C]{c}{
		escapeinside=@@,
		tabsize=4,
		linenos
	}

	\newminted[Caml]{ocaml}{
	escapeinside=@@,
	tabsize=4
}
	
% OTHER
	
	\renewcommand{\thealgocf}{}
	\SetInd{1em}{1em}
	
	\newcommand{\entspace}{\hspace*{17.4mm}\ignorespaces}
	\newcommand{\aentspace}{\hspace*{17.4mm}\ignorespaces}
	% \newcommand{\hypspace}{\hspace*{18.5mm}\ignorespaces}
	\newcommand{\listspace}{\hspace*{2.6mm}\ignorespaces}
	\newcommand{\listskip}{\hspace*{5mm} \listspace\ignorespaces}
	
	\renewcommand{\mod}[1]{\,\,\, [\text{mod }#1]}
	
	\newcommand{\codecom}[1]{\colsep{1.2pt}\quad{\footnotesize \textbf{/\!\!/\emph{#1}}}}
	
% OUTDATED

	\newcommand{\plabel}[1]{\textbf{#1}}
	
	\newcommand{\uplabel}[1]{\underline{\textbf{#1}}}
	
	\newcommand{\thuplabel}[2]{
		\vspace{4mm}
		\begin{thmbox}[S,leftmargin=8mm]{\uplabel{#1}}
			{#2}
		\end{thmbox}
		\vspace{2mm}
	}
	
	\newcommand{\shortproof}[1]{\plabel{\textcolor{lightgray}{$\bm{\blacklozenge}$} \textcolor{darkgray}{Preuve : }}{\small \textcolor{darkgray}{#1} \normalsize \nl}}
	
	\newcommand{\corr}[1]{\plabel{\textcolor{lightgray}{$\bm{\blacklozenge}$} \textcolor{darkgray}{Correction :}} \\[2mm] \hspace*{3.5mm} \parbox{167mm}{#1} \nl}
\usepackage{euscript}
\DeclareMathOperator{\argormax}{(arg)max}
\DeclareMathOperator{\val}{val}

\begin{document}
	
\title{Algorithmique du texte}

\intro{On appelle texte toute suite finie de caractères, c'est-à-dire ce qu'on a appelé ``mot'' jusqu'à présent. L'algorithmique du texte consiste à résoudre des problèmes sur des textes qui peuvent en \\ réalité modéliser des informations diverses : textes réels, séquences d'ADN, musique, images...}

L'algorithmique du texte présente des applications dans la résolution de nombreux problèmes en informatique, par exemple : \\
	\hs{5} \bdot les recherches de similarité (plus long sous-mot commun, plus long facteur commun, distance\\ \hs{7.8} d'édition et recherche d'alignements...) \\
	\hs{5} \bdot la recherche de motifs (penser au \textsf{Ctrl\,+\,F} dans un éditeur de texte ou un navigateur) \\
	\hs{5} \bdot la compression de texte (encodage des caractères sur un nombre variable de bits, encodage par \\ \hs{7.8} facteurs...) \nt
	%
Dans ce chapitre, nous traiterons et résoudrons quelques uns de ces problèmes.

\eqskip{3mm}
\vs{1}
\begin{Rappel}[alphabets, mots et concaténation]
	On rappelle qu'un alphabet est un ensemble de fini et non vide de caractères. Pour \(\Sigma\) un alphabet, on définit l'ensemble des mots sur \(\Sigma\) et l'ensemble des mots \emph{non vides} sur \(\Sigma\), notés respectivement \(\Sigma^*\) et \(\Sigma^+\), par :
		\begin{align*}
			& \text{\bdot} \Sigma^* = \bigcup_{n\in\bb{N}} \Sigma^n \\
			& \text{\bdot} \Sigma^+ = \bigcup_{n\in\bb{N}^*} \Sigma^n = \Sigma^*\backslash\{\varepsilon\}
		\end{align*}
	où \(\varepsilon\) désigne le mot vide, unique élément de \(\Sigma^0\). \nt
		%
	\colsep{1.5pt}
	\eqskip{2mm}
	On définit aussi l'opération de concaténation sur \(\Sigma^*\), notée ``\(\cdot\)'' :
		\[
			\forall\left(\!(u_i)_{i\in[1..n]}, (v_j)_{j\in[1..m]}\right)\in(\Sigma^*)^2, \,u\cdot v = (w_k)_{k\in[1..n+m]} \text{ où}\,
			\left\{ \begin{tabular}[h]{l}
				\(\forall\,k\in[1..n]\), \(w_k = u_k\) \\
				\(\forall\,k\in[n+1..n+m]\), \(w_k = v_{k-n}\)
			\end{tabular} \right.
		\]
	Alors, \((\Sigma^*,\cdot)\) est un monoïde (\emph{i.e.} la loi \(\cdot\) est interne associative et possède un élément neutre). Son neutre est \(\varepsilon\).
\end{Rappel}

\begin{Rappel}[sous-mots, facteurs, préfixes et suffixes]
	Soit \((u,v)\in(\Sigma^*)^2\), on note \(n=|u|\) et \(m=|v|\). Alors : \\
		\hs{5} \bdot \(u\) est un sous-mot de \(v\) ssi il existe une application \(\varphi\in\EuScript{F}([1..n],[1..m])\) strictement \\ \hs{7.8} croissante telle que \(u = (v_{\varphi(i)})_{i\in[1..n]}\) \\
		\hs{5} \bdot \(u\) est un facteur de \(v\) ssi il existe \(i\in[1..m-n]\) tel que \(u = (v_{i+k})_{k\in[1..n]}\) \\
		\hs{5} \bdot \(u\) est un préfixe de \(v\) ssi il existe \(w\in\Sigma^*\) tel que \(v = u\cdot w\) \\
		\hs{5} \bdot \(u\) est un suffixe de \(v\) ssi il existe \(w\in\Sigma^*\) tel que \(v = w\cdot u\).
\end{Rappel}

\vs{2}
\begin{Remarque}
	De façon alternative, \(u\in\Sigma^*\) est un facteur de \(v \in \Sigma^*\) ssi il existe \((i,j)\in[1..|u|]^2\) tel que \(u = (v_k)_{k\in[i..j]}\). En particulier, \(\varepsilon\) est toujours un facteur de \(v\).
\end{Remarque}

\pagebreak
\section{Plus long facteur commun}
	
	On considère le problème \textsf{Plus long facteur commun}, que l'on abrège en \textsf{PLFC} :
	\begin{center}
		\pbm{PLFC}{\(u\in\Sigma^*\) de longueur \(n\) \\ \entspace \(v\in\Sigma^*\) de longueur \(m\) \\[-3mm]}{\(\argormax \left\{ |f| \,\middle|\, f\in\Sigma^*, \hs{-2}
		\begin{tabular}[h]{l}
			\(\exists\,(i,j)\in[1..n]^2,\, f = (u_k)_{k\in[i..j]}\) \\
			\(\exists\,(p,q)\in[1..m]^2,\, f = (v_k)_{k\in[p..q]}\)
		\end{tabular} \hs{-2}\right\}\)}
	\end{center}
	\vs{2}
	\eqskip{2mm}
	On en propose une résolution par programmation dynamique. \\
	Soit \((u,v)\in(\Sigma^*)^2\). Notons \(n=|u|\) et \(m=|v|\). \nt
	Pour \((i,j)\in[0..n]\times[0..m]\), on pose :
			\[
				A_{i,j} = \max\{|s| \,|\, \text{\(s\) est un suffixe de \(u_1...u_i\) et \(v_1...v_j\)}\} \leq \min(i,j)
			\]
	La propriété qui suit donne alors une façon de résoudre le problème à partir de ces sous-problèmes.
			
	\begin{Proprietes}
		\colsep{1.5pt}
		On a :\!
			\begin{tabular}[t]{cl}
				\i & \(A_{0,0} = 0\) \\
				\ii & \(\forall\,i\in[0..n]\), \(A_{i,0} = 0\) \\
				\iii & \(\forall\,j\in[0..m]\), \(A_{0,j} = 0\) \\[-2mm]
				\iv & \(\forall\,(i,j)\in[1..n]\times[1..m]\), \(A_{i,j} =
				\left\{ \begin{tabular}[h]{l}
					\(A_{i-1,j-1} + 1\) si \(u_i = v_j\) \\
					\(0\) sinon
				\end{tabular} \right.\)
			\end{tabular} \nt
		De plus, \(\mathsf{PLFC}(u,v) = \displaystyle \max_{(i,j)\in[0..n]\times[0..m]} A_{i,j}\).
	\end{Proprietes}


\section{Recherche de motifs}

	On s'intéresse à présent à la recherche de motifs dans un texte. Plus précisément, on veut obtenir toutes les occurences d'un motif donné dans un texte donné, ce qui peut se formaliser comme suit.
		\begin{center}
			\pbm{RDM}{\(t\in\Sigma^*\) un texte de longueur \(n\) \\ \entspace \(x\in\Sigma^*\) un motif de longueur \(m\)}{\(\left\{i\in[1..n-m+1] \,|\, (t_{i+k})_{k\in[0..m-1]} = x\right\}\)}
		\end{center}
	Remarquons que dans cette formalisation du problème, on obtient en sortie de \(\mathsf{RDM}(t,x)\) l'ensemble des indices de \emph{début} des occurences de \(x\) dans \(t\).
	
	\subsection{Algorithme naïf}
	
		Une première approche possible pour sa résolution est celle naïve que propose l'algorithme suivant.
		
		\begin{algo}{Motif\_naïf}{$(t_i)_{i\in[1..n]} \in \Sigma^*$ et $(x_i)_{i\in[1..m]} \in \Sigma^*$}{}{}
			\(i = 1\) \\
			Tant que \(i \leq n-m+1\) \\ \Indp
				\(j = 0\) \\
				Tant que (\(j<m\) et \(t_{i+j} = x_{j+1}\)) \\ \Indp
					\(j \gets j+1\) \\ \Indm
				Si \(j = m\) alors \\ \Indp afficher \(i\) \\ \Indm
				\(i \gets i+1\)
		\end{algo}
	
		\textsf{Motif\_naïf} est facilement en \(\Theta(m(n-m))\), soit encore \(\Theta(nm)\) quand \(|x| \ll |t|\). Cependant, on peut faire mieux : c'est précisément l'objet de la prochaine propriété, qui introduit un principe reposant sur la représentation des entiers en base \(b\) et permettant d'écrire des algorithmes plus efficaces.
	
	\subsection{Algorithmes de Rabin-Karp}
	
		\vs{-2}
		\begin{Propriete}
			Soit \(\Sigma = [0..b[\). Pour \(w \in \Sigma^m\) et \((g,d)\in\Sigma^2\), on a :
				\begin{align*}
					& \text{\bdot} \val_b(g\cdot w) = gb^m + \val_b(w) \\[-1mm]
					& \text{\bdot} \val_b(w\cdot d) = b\val_b(w) + d = b (\val_b(gw) - gb^m) + d
				\end{align*}
		\end{Propriete}
		
		\vs{2}
		\begin{algo}{Rabin-Karp}{\(t\in\Sigma^*\) où \(|\Sigma|=b\) \\ \aentspace \(x\in\Sigma^*\) avec \(|x| \leq |t|\)}{}{}
			\(n,m = |t|, |x|\) \\
			\(c_x, c_f, b_m = 0,0,1\) \\
			Pour \(i\) allant de \(1\) à \(m\) \\ \Indp
				\(c_x \gets (c_x * b) + x_i \) \\
				\(c_f \gets (c_f * b) + t_i\) \\
				\(b_m \gets b \times b_m\) \\ \Indm
			Si \(c_f = c_x\) alors \quad{\footnotesize \textbf{/\!\!/\emph{ici \(\bm{b_m = b^m}\)}}} \\ \Indp afficher 1 \\ \Indm
			Pour \(d\) allant de 0 à \(n-m-1\) \\ \Indp
				\(c_f \gets b(c_f - t_{d+1}b_m) + t_{d+m+1}\) \\
				Si \(c_x = c_f\) alors \\ \Indp
					afficher \((d+2)\)
		\end{algo}
		
		\bdot Si l'on compte les additions et les multiplications, la première boucle ``Pour'' est en \(\Theta(m)\). \\ 
		\bdot De même, il y a de l'ordre de \(\Theta(n-m)\) tours de la deuxième boucle ``Pour'', chacune faisant intervenir des opérations \(=,+,-,\times\) qui sont en \(\Theta(m)\). \nt
			%
		D'où une complexité totale en \(\Theta(nm)\) : elle n'est donc pas pour l'instant meilleure que \textsf{Motif-naïf}. \nll
			%
		Afin de l'améliorer, on modifie l'algorithme en décidant de réaliser cette fois les opérations modulo un entier \(q > 0\), ce qui permet de rendre leurs complexités constantes (c'est-à-dire en \(\Theta(1)\)).
		
		\begin{algo}{Rabin-Karp\_bis}{\(q\in\bb{N}^*\) \\ \aentspace \(t\in\Sigma^*\) avec \(|\Sigma|=b\) \\ \aentspace \(x\in\Sigma^*\) avec \(|x| \leq |t|\)}{}{}
			\(n,m = |t|, |x|\) \\
			\(c_x, c_f, b_m = 0,0,1\) \\
			Pour \(i\) allant de \(1\) à \(m\) \\ \Indp
				\(c_x \gets (c_x * b) + x_i \mod{q}\) \\
				\(c_f \gets (c_f * b) + t_i \mod{q}\) \\
				\(b_m \gets b \times b_m\) \\ \Indm
			Si \(c_f = c_x\) alors \\ \Indp
				Si \(t_1...t_{m} = x\) alors \\ \Indp
					afficher \(1\) \\ \Indm \Indm
		\end{algo}
		
		\begin{algocont}
			Pour \(d\) allant de \(0\) à \(n-m-1\) \\ \Indp
			\(c_f \gets b(c_f - t_{d+1}b_m) + t_{d+m+1} \mod{q}\) \\
			Si \(c_f = c_x\) alors \\ \Indp
			Si \(t_{d+2}...t_{d+m+1} = x\) alors \\ \Indp
			afficher \((d+2)\) \\ \Indm \Indm
		\end{algocont}
		
		Cette nouvelle version de l'algorithme présente toujours des tests d'égalité en \(\Theta(m)\) dans les boucles ``Pour'', mais ils se produisent maintenant beaucoup plus rarement, ce qui affaiblit la complexité moyenne.
		
	\subsection{Algorithmes de Boyer-Moore}
		
		On donne enfin deux exemples d'algorithmes qui exploitent la forme même du motif afin d'effectuer une recherche plus intelligente de ses éventuelles occurences dans le texte considéré.
		
		\subsubsection{Algorithme de Boyer-Moore-Horspoole}
		
			\vs{-2}
			\colsep{2.2pt}
			\begin{Notation}
				Pour \(x\in\Sigma^m\), on pose :
					\(
						f_x = \fun{\Sigma}{\bb{N}}{a}{
							\left\{ \begin{tabular}[h]{l}
								\(m\) si \(a\notin\{x_i \,|\, i\in[1..m-1]\}\) \\
								\(m - \max\left\{i\in[1..m-1] \,|\, x_i = a\right\}\) sinon
							\end{tabular} \right.
						}
					\)
			\end{Notation}
			
			\vs{2}
			\begin{Remarque}
				Si \(a\in\Sigma\), on peut aussi écrire \(f_x(a) = m - \max\left(\left\{i\in[1..m-1]\,|\,x_i = a\right\} \cup \{0\}\right)\).
			\end{Remarque}
			\vs{2}
			
			\begin{algo}{Precalcul\_BMH}{\(x\in\Sigma^*\)}{}{}
				\(f = \text{tableau indexé par \(\Sigma\), initialisé à \(m\)}\) \\[-2mm]
				Pour \(i\) allant de \(1\) à \(m-1\) \colsep{1.2pt}\quad{\footnotesize \textbf{/\!\!/\emph{\(\bm{\,\forall\,a\in\Sigma,\, f[a] =
					\left\{}\begin{tabular}[h]{l}
						\(\bm{m}\) si \(\bm{a\notin \{x_j \,|\, j\in[1..i-1]}\}\) \\
						\(\bm{m - \max\left\{ j \in [1..i-1] \,|\, x_j = a\right\}}\) sinon
					\end{tabular} \right.\)}}} \\[-3mm] \Indp
					\(f[x_i] \gets m-i\) \\ \Indm
				Retourner \(f\)
			\end{algo}
		
			\begin{algo}{Boyer-Moore-Horspoole}{\(x,t\in\Sigma^*\)}{}{}
				\(n,m = |t|,|x|\) \\
				\(f = \mathsf{Precalcul\_BMH}(x)\) \\
				\(d=0\) \\
				Tant que \(d\leq n-m\) \\ \Indp
					Si \(t_{d+1}...t_{d+m} = x_1...x_m\) alors \\ \Indp
						afficher \(d+1\) \\ \Indm
					\(d \gets d+f[t_{d+m}]\)
			\end{algo}
			
			\vs{4}
			\begin{Exemple}
				\emph{cf.} annexe.
			\end{Exemple}
		
		\subsubsection{Algorithme simplifié de Boyer-Moore}
			
			\pagebreak
			\colsep{2.2pt}
			\begin{Notation}
				Pour \(x\in\Sigma^m\), on note :
				\(
				g_x = \fun{\Sigma}{\bb{N}}{a}{
					\left\{ \begin{tabular}[h]{l}
						\(m\) si \(a\notin\{x_i \,|\, i\in[1..m]\}\) \\
						\(m - \max\left\{i\in[1..m] \,|\, x_i = a\right\}\) sinon
					\end{tabular} \right.
				}
				\)
			\end{Notation}
		
			\begin{algo}{Precalcul\_BMS}{\(x\in\Sigma^*\)}{}{}
				\(g = \text{tableau indexé par \(\Sigma\), initialisé à \(m\)}\) \\
				Pour \(i\) allant de \(1\) à \(m\) \\ \Indp
				\(f[x_i] \gets m-i\) \\ \Indm
				Retourner \(g\)
			\end{algo}
		
			\begin{algo}{Boyer-Moore\_simplifié}{\(x,t\in\Sigma^*\)}{}{}
				\(n,m = |t|,|x|\) \\
				\(g = \mathsf{Precalcul\_BMS}(x)\) \\
				\(d=0\) \\
				Tant que \(d\leq n-m\) \\ \Indp
					\(i = 0\) \\ 
					Tant que (\(t_{d+m-i}=x_{m-i}\) et \(i < m\)) \\ \Indp
						\(i \gets i+1\) \\ \Indm
					Si \(i=m\) alors \\ \Indp
						afficher \(d+1\) \\
						\(d \gets d+1\) \\ \Indm
					Sinon \\ \Indp
						\(d \gets d+\max(1,g[t_{d+m-i}]-i)\)
			\end{algo}
			
			\vs{4}
			\begin{Exemple}
				\emph{cf.} annexe.
			\end{Exemple}
		
\section{Compression}

	\'Etant donné un texte \(t = t_1t_2...t_k...t_n\) dont les caractères sont dans un alphabet \(\Sigma\), on cherche à l'encoder par des caractères d'un ``sur-alphabet'' ou ``super-alphabet'' \(\widehat{\Sigma}\) dont certains caractères encoderont des facteurs de \(t\), ceci dans le but de réduire la taille de stockage du texte. \nt
		%
	On aura donc un application \(\varphi\in\EuScript{F}_p(\Sigma^+,\widehat{\Sigma})\) telle que \(\Sigma \subseteq \EuScript{D}(\varphi)\), injective (voire même bijective, quitte à restreindre \(\widehat{\Sigma}\)).
	
	\vs{-2}
	\begin{Illustration}
		Concrètement, pour un texte \(t = \overbrace{t_1...t_{r_1}|t_{r_1+1}...t_{r_2}|\ldots|t_{r_{K-1}+1}...t_{n}}^K\), on aura un encodage de la forme \(c = \varphi(t_1...t_{r_1})\,\varphi(t_{r_1+1}...t_{r_2})\ldots\varphi(t_{r_{K-1}+1}...t_n) \in \widehat{\Sigma}^K\).
	\end{Illustration}
	
	\subsection{Algorithmes de Lempel-Ziv-Welch}
	
	Voyons dans un premier temps comment on peut intuitivement réaliser la compression et la décompression d'un texte, à partir d'un exemple bien choisi.
	
	\vs{2}
	\begin{Exemple}
		Prenons \(t = \text{\textsf{AUTOAUTOTAU}}\) sur \(\Sigma_\ell\), l'ensemble des lettres de l'alphabet latin majuscule.
	\end{Exemple}
	
	\vs{2}
	On donne maintenant les pseudo-codes correspondant aux deux algorithmes que nous venons d'appliquer, appelés algorithmes de Lempel-Ziv-Welch.
	\vs{2}
	
	\begin{algo}{Comp\_LZW}{\(t = (t_i)_{i\in[0..n]}\) un texte \\ \aentspace \(d\) un dictionnaire}{}{$t\in\Sigma^+$, les clés de \(d\) sont exactement \(\Sigma\) et ses valeurs sont dans $[0..|\Sigma|-1]$}
		\(n,k = |t|, |\Sigma|\) \\
		\(\rm{res} = \varepsilon\) \\
		\(m = t[\,0\,]\) \\
		Pour \(i\) allant de \(1\) à \(n-1\) \\ \Indp
			Si \(m\cdot t[\,i\,] \in d.\text{\textsf{clés}}\) alors \\ \Indp
				\(m \gets m\cdot t[\,i\,]\) \\ \Indm
			Sinon \\ \Indp
				\(\rm{res} \gets \rm{res} \cdot d[m]\) \\
				\(d.\text{\textsf{ajouter}}\,(\overbrace{m\cdot t[\,i\,]}^{\rm{clé}}, \!\overbrace{k}^{\rm{valeur}}\!)\) \\
				\(k \gets k+1\) \\
				\(m \gets t[\,i\,]\) \\ \Indm \Indm
		Retourner \(\rm{res}\)
	\end{algo}
	
	\vs{4}
	\begin{algo}{Décomp\_LZW}{\(c = (c_i)_{i\in[0..n-1]}\) un texte non vide \\ \aentspace \(d\) un dictionnaire}{}{les clés de \(d\) sont $[0..|d|-1]$}
		\(n,k = |c|, |d|\) \\
		\(\rm{res} = d[\,c[0]\,]\) \\
		\(m = d[\,c[0]\,]\) \\
		Pour \(i\) allant de \(1\) à \(n-1\) \\ \Indp
			Si \(c[\,i\,] \in d.\text{\textsf{clés}}\) alors \\ \Indp
				\(r = d[\,c[i]\,]\) \\ \Indm
			Sinon \\ \Indp
				\(r = m \cdot m[\,0\,]\) \\ \Indm
			\(\rm{res} \gets \rm{res} \cdot r\) \\
			\(d.\text{\textsf{ajouter}}\,(\overbrace{k}^{\text{clé}}, \!\overbrace{m\cdot r[\,0\,]}^{\text{val. \!associée}} \!)\) \\
			\(k \gets k+1\) \\
			\(m \gets r\) \\ \Indm
		Retourner res
	\end{algo}

\end{document}